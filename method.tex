\chapter{Method}\label{cha:method}
This chapter will give a detailed description on which how the problem real-time text rendering on mobile devices was solved in this thesis. This chapter is divided into two different parts. The first part gives a detailed description about the implementation of the distance transform module. This part has been reimplemented several times due to poor performance. The second part gives a detailed description about the implementation of the distance field rendering module.

\section{Distance field generation initial attempt}
There is many EDT algorithms for creating distance fields. Most of the EDT algorithms used today run in $\mathcal{O}(nm)$, where n and m are the width respectively the height of the image. Example of algorithms running in $\mathcal{O}(nm)$ is \citet{Danielsson} and \citet{meijster}.

The initial implementation of distance field generation was built around \citet{Gustavson:2011} article on the subject. A version of the 8SED algorithm was used along with the anti-aliased sub-pixel distance measure proposed in the article. A signed distance field has several advantages compared to an unsigned distance field. The most obvious advantages is the increased flexibility it gives and that it is easier to perform proper anti-aliasing on the border due to the increased information in an environment around the border pixels\citep{gustavson20122d}. This is more than enough reason to implement the generation to generate a signed distance field. To generate a signed distance field the distance transform was ran twice with inverted input representation. This will make one of the transforms create a distance field on the inside of the glyph and the other transform on the outside of the glyph. The resulting distance field was then calculated with the following formula.\vspace{\baselineskip}\newline
$result(i,j) = inside(i,j) - outside(i,j), \forall (i, j), i \in \{0,\dots, m-1\}, j \in \{0,\dots, n-1\}$ \vspace{\baselineskip}\newline
