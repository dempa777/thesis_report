\chapter{Introduction}\label{cha:intro}
It is hard to create a text rendering solution that is fast, dynamic and outputs good looking text. Distance field text rendering was introduced for the public in an article from valve 2007 and is used in many applications today. The goal with this thesis was to evaluate the distance field text rendering and implement it in a cross-plattform development tool used by VisiArc.
\section{Problem formulation}
This section covers the problem formulations which were set up at the beginning of this thesis. The problem formulation has functioned as guidelines throughout the work of the thesis.
\begin{itemize}
  \item How can distance fields be used when rendering text and how does the technique work?
  \item What distance field size is it necessary to use for the rendered text to get equally good appearance as the text rendered from Visiarcs current text rendering implementation?
  \item Is it possible to generate a distance field fast enough for the user not to be able to determine if the distance field was pre generated or not?
\end{itemize} 
\section{Structure}
This chapter described the goal and the problem formulation of this thesis. The next chapter gives basic knowledge to facilitate further reading of the report. Chapter 3 includes information about related work that has helped the implementation of this thesis. Chapter 4 and 5 describes the methods used for implementing respectively evaluating the text rendering solution of this thesis. The methods for evaluation was used to produce the result presented in chapter 6 and discussed in chapter 7. Chapter 8 which is the last chapter of this report proposes some ideas for future work within the scope of this thesis.