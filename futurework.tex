\chapter{Future work}\label{cha:conclusions}
The focus of this thesis has been around generating distance fields fast that produces good looking text. There are other factors that matters when rendering text to the screen. For example how the glyphs are stored in textures for the GPU to be able to render the text fast. The implementation of this thesis is far from optimal when it comes to the rendering. This is something that needs more work. The implementation of text rendering done in this thesis renders every character with a separate draw call to the GPU on a separate quad. For example, a text containing 10000 characters will perform 10000 draw calls per frame and 40000 vertices for the GPU to work with.

A good thing with distance fields are the possibility to do text effects. For example a red border can easily be added to a glyph by setting the color red in the fragment shader if the interpolated distance value in that pixel is in the range [0.5,0.6]. Another possible effect is to use some kind of function over the boundry between the inside and the outisde of the glyph. This could create cool effects with for example sinus outline of the glyph. To create even better and more living effects the current time could be sent to the fragment shader and used in the fragment shader calculations creating for example moving waves over the outlines of a glyph or a glyph with pulsating size.